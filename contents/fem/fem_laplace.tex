\section{Model problem}
It is instructive to begin at a simple level and proceed by incrementally adding to the
complexity of the equations we are discretising when explaining the use of the FEM, so
we begin by considering the classical heat equation: given $T>0$, for $t \in[0,T]$ find $u(t,x)$ such that
\begin{subequations}
\begin{align}
\label{eqn:heat_strong}
\frac{\partial u}{\partial t} - \nabla \cdot \nabla u = 0\;\;\; \mbox{in} \; \Omega_{t},\\
\boldsymbol{n} \cdot \nabla u = {g}_{N}   \;\;\; \mbox{on}\; \Gamma_{N},
\\
\label{eqn:heat_dirichlet}
 u = {g}_{D}   \;\;\; \mbox{on}\; \Gamma_{D},
\\
 u(0,x) = u^{0}(x)   \;\;\; \mbox{in}\; \Omega.
\end{align}
\label{eqn:laplace}
\end{subequations}
Here $\Omega$ is a bounded domain in $\mathbb{R}^{2}$ or $\mathbb{R}^{3}$, with boundary $\partial\Omega = \Gamma_{N}  \cup \Gamma_{D} $, that has an outward pointing unit normal $\boldsymbol{n}$. The initial condition is given by $u^{0}(x)$. In the case where ${g}_{N}= 0$, system (\ref{eqn:laplace}) can describe the evolution of heat in an
object with geometry described by $\Omega$, where we have perfect thermal insulation on $\Gamma_{N}$ and fixed temperature distributions given by the function ${g}_{D}$ defined on the boundary due to some part of the environment with fixed temperature contacting the object along $\Gamma_{D}$.

\subsection{Weak formulation}

The strong form of (\ref{eqn:laplace}) requires $u$ to be at least twice differentiable. To weaken the regularity restrictions we multiply equation (\ref{eqn:heat_strong}) by an arbitrary function $v$, called a test function, and integrate over $\Omega$: 
\begin{equation*}
\left( \frac{\partial u}{\partial t},v \right)  -\left(\nabla \cdot \nabla u,v \right) = 0.
\end{equation*}
Applying the divergence theorem, this equation can be rewritten:
\begin{multline*}
\left( \frac{\partial u}{\partial t},v \right)  -\left(\nabla u \cdot \normal ,v \right)_{\partial \Omega}+ \left(\nabla u,\nabla v \right)\\= \left( \frac{\partial u}{\partial t},v \right)  -\left(\nabla u \cdot \normal ,v \right)_{\Gamma D}-\left(g_{N} ,v \right)_{\Gamma N}+ \left(\nabla u,\nabla v \right)=0.
\end{multline*}
Here $\left(\cdot ,\cdot\right)_{\Gamma_{N}}$ and $\left(\cdot ,\cdot\right)_{\Gamma_{D}}$ denote the inner product taken over $\Gamma_{N}$ and $\Gamma_{D}$, respectively. Taking note of the Dirichlet condition (\ref{eqn:heat_dirichlet}), and letting $v = 0$ on $\Gamma_{D}$, we arrive at the following equation:
\begin{equation*}
\left( \frac{\partial u}{\partial t},v \right) + \left(\nabla u,\nabla v \right)=\left(g_{N} ,v \right)_{\Gamma N}.
\end{equation*}
Note that in this equation the second derivatives of $u$ need not exist. With that in mind, both the solution and the test functions can come from the space $H^1(\Omega)$, as long as they satisfy the appropriate Dirichlet boundary conditions. For convenience we will use the notation $X_{D}=\left\lbrace v \in H^{1}(\Omega) | v = u_{D} \;\mbox{on} \; \Gamma_{D} \right\rbrace$ and $X_{0}=\left\lbrace v \in H^{1}(\Omega) | v = 0 \;\mbox{on} \; \Gamma_{D} \right\rbrace$. The weak formulation of (\ref{eqn:heat_strong}) is as follows: Find $u\in X_{D}$ such that
\begin{equation}
\left( \frac{\partial u}{\partial t},v \right) + \left(\nabla u,\nabla v \right)=\left(g_{N} ,v \right)_{\Gamma N}\;\;\forall v \in X_{0}.
\label{eqn:heat_weak}
\end{equation}

\subsection{Time discretisation}

We also need to choose a method of treating the time derivative. In this work, we do so using Euler difference quotients, and so we make the approximation $u_{t}(t+\Delta t,x) \approx \frac{u(t+\Delta t,x) - u(t,x)}{\Delta t}$ for some constant time step $\Delta t$. We write $u(x)^{n}$ for the the temporally-semidiscrete approximation to $u(n\Delta t,x)$, and our numerical scheme will yield approximations at times $t=0,\Delta t,2\Delta t,...,T$. Inserting this difference quotient and assuming that $\Delta T$ divides $T$, equation (\ref{eqn:heat_weak}) becomes: for $n=1,2,...,\frac{T}{\Delta t}$, find $u^{n}\in X_{D}$ such that
\begin{equation}
\left( u^{n},v \right) + \Delta t \left(\nabla u^{n},\nabla v \right)=\left(g_{N} ,v \right)_{\Gamma N}+\left( u^{n-1},v \right) \;\;\forall v \in X_{0}.
\label{eqn:heat_weak}
\end{equation}
%
\subsection{Finite element discretisation}
In order to solve this problem numerically, we must make it finite dimensional by discretising it suitably. The finite element approximation space is constructed as follows: first, the problem domain is partitioned into small element domains, and second, the element is defined by prescribing for each element domain a set of nodes and nodal values, and defining  suitable basis functions on these, for example, as piecewise-linear basis functions. 

Element domains are normally shaped as triangles or squares in $\mathbb{R}^{2}$, tetrahedra or hexahedra in $\mathbb{R}^{3}$. All the nodes, edges and faces of element domains constitute
the problem mesh. Defining local basis functions completes the finite element space. For a rigorous definition of finite elements, and a description of different types of elements we refer to \citet{brenner2008mathematical}. 

Let $\mathcal{T}^{h}$ be a partition of $\Omega$ into non-overlapping elements $K$. We denote by $h$ the size of the largest element in $\mathcal{T}^{h}$. On the given partition $\mathcal{T}^{h}$  we then define the following finite element spaces, to solve the model problem:
%
\begin{equation*}
X_{hD}=\left\lbrace u  \in C^{0}(\Omega) : u |_{K} \in P_{1}(K); u  = u_{D} \;\mbox{on} \;\Gamma_{D} ; \forall K \in \mathcal{T}^{h} \right\rbrace,
\label{eqn:fespace_modelD}
\end{equation*}
\begin{equation*}
X_{h0}=\left\lbrace u  \in C^{0}(\Omega) : u |_{K} \in P_{1}(K); u  = 0 \;\mbox{on} \;\Gamma_{D} ; \forall K \in \mathcal{T}^{h} \right\rbrace,
\label{eqn:fespace_model0}
\end{equation*}
where  $P_{1}(K)$ is the space of linear polynomials on $K$, and $C^{0}(\Omega)$ is the space of continous functions on $\Omega$. The discretised problem, for each time step, is to find $u^{n}_{h}\in X_{hD}$ such that 
\begin{equation}
\left( u^{n}_{h},v_{h} \right) + \Delta t \left(\nabla u_{h}^{n},\nabla v_{h} \right)=\left(g_{N} ,v_{h} \right)_{\Gamma N}+\left( u^{n-1}_{h},v_{h} \right) \;\;\forall v_{h} \in X_{h0}.
\label{eqn:heat_weak_disc}
\end{equation}
We now choose the Lagrangian basis $\left\lbrace \phi_{1},\phi_{2},...,\phi_{m} \right\rbrace$ of $X^{h}$ defined by the nodal values at the nodes $\left\lbrace \mathbf{x}_{1},\mathbf{x}_{2},...,\mathbf{x}_{m} \right\rbrace$, namely
%
\begin{equation*}
\phi_{i}(\mathbf{x}_{j}) =\delta_{i,j}= \left\lbrace
  \begin{array}{l l}
    1,\;\;i=j\\    
    0,\;\;i \neq j
  \end{array} \right. ,
\end{equation*}
%
%
We observe that a basis of $X_{h0}$ can be constructed by removing $\phi_{i}$ with $\mathbf{x}_{i}\in \Gamma_{D}$ from the basis of $X_{h}$. Let us assume that the indices of such basis functions are
$1,...,m,$ and therefore $X_{h0} = \mbox{span}\left\lbrace \phi_{1},...,\phi_{m} \right\rbrace$. The finite-dimensional weak
problem (\ref{eqn:heat_weak_disc}) is equivalent to: Find $u^{n}_{h}\in X_{hD}$ such that 
\begin{equation}
\left( u^{n}_{h},\phi_{i} \right) + \Delta t \left(\nabla u^{n}_{h},\nabla \phi_{i} \right)=\left(g_{N} ,\phi_{i} \right)_{\Gamma N}+\left( u^{n-1}_{h},\phi_{i} \right) \;\;\forall i=1,...,m.
\label{eqn:heat_weak_basis}
\end{equation}
Any function from $X_{h}$ can be presented in the form of a basis expansion. Let this basis expansion for $u_{h}$ be
\begin{equation*}
u_{h}^{n}=\sum^{m}_{i=1} u_{i}^{n}\phi_{i},
\end{equation*}
with $u_{i}^{n}=u_{h}^{n}(\mathbf{x}_{i})$. We define the vector of nodal values to be $\mathbf{u}^{n} = \left[u_{1}^{n},...,u_{m}^{n} \right]^{T}$. Substituting this expression into (\ref{eqn:heat_weak_basis}), we finally obtain a linear system which we can solve for $\mathbf{u}^{n}$:
\begin{equation}
(\mathbf{M}+\Delta t \mathbf{A})\mathbf{u}^{n}=\mathbf{M}\mathbf{u}^{n-1} + \mathbf{g},
\label{eqn:heat_linear_system}
\end{equation}
 where we have defined the following matrices and vectors:
 \begin{equation*}
  \mathbf{A}=[\mathbf{a}_{ij}], \;\; \mathbf{a}_{ij}=\int_{\Omega} \nabla \mathbf{\phi}_{i} \cdot \nabla\mathbf{\phi}_{j}\,\mbox{d}x,
 \end{equation*}
 \begin{equation*}
  \mathbf{M}=[\mathbf{m}_{ij}], \;\; \mathbf{m}_{ij}=\int_{\Omega}  \mathbf{\phi}_{i} \cdot \mathbf{\phi}_{j}\,\mbox{d}x, 
 \end{equation*}
  \begin{equation*}
  \mathbf{g}=[\mathbf{g}_{i}], \;\; \mathbf{g}_{i}=\int_{\Gamma_{N}}{{g}_{N}} \cdot {\phi}_{i}\,\mbox{d}s, 
 \end{equation*}
%\begin{equation*}
%  \mathbf{u}^{n}=(u_{1}^{n},...,u_{m}^{n})^{T}. 
% \end{equation*}
%Matrices $\mathbf{A}$ and $\mathbf{M}$ are often referred as the stiffness and mass matrices, with terminology from early applications of FEM in structural mechanics.
%
The linear system of equations (\ref{eqn:heat_linear_system}) can be solved using standard methods such as Gaussian elimination.







