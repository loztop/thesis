
\section{Introduction}
%Miguel
A large proportion of the mathematical models in science and engineering take the
form of differential equations. Only in the simplest cases, or under strong assumptions,
is it possible to find exact analytical solutions to the equations in the model.
%Often, one has to rely on numerical techniques for finding approximate solutions
%for particular parameter sets using computers. 
%The finite element method is a general
%technique for the numerical solution of differential equations. 
%
%The method was introduced by engineers in the late 1950's and early 1960's for the numerical
%solution of PDEs in structural engineering. When the mathematical study
%of the finite element method started in the mid 1960's it soon became clear that in
%fact the method is a general technique for numerical solution of partial differential
%equations with roots in the variational methods in mathematics introduced at the
%beginning of the 20th century.
%
%
%
%
%Lyia
Numerical methods are an established means of solving differential equations that are
of practical interest in a variety of applied problems. Finite difference, finite volume
and finite element methods are the most widely used types of such methods. Their
basic idea is replacing the original infinite-dimensional problem by a finite-dimensional
approximation, which is, generally speaking, easier to solve.
%
%The main idea of finite difference methods is to approximate the derivatives appearing
%n the equations with finite differences at a set of discrete points in the domain
%of interest. Their advantages include relative simplicity of derivation and implementation.
%However, finite difference methods are best suited for simple domain shapes
%and regular meshes, while using them for problems on complex geometries requires
%non-standard treatment even for standard differential equations. For an introduction
%to finite differences in partial differential equations we refer to [90, 116, 114].
%Finite volume methods [87, 121] rely on the application of the divergence theorem
%within small subdomains of the original domain – finite volumes – and approximating
%fluxes through volume boundaries. These methods are traditionally used in the
%field of computational fluid dynamics due to their conservative properties, although
%9
%formulating problems on complex geometries is again far from trivial.
Finite element methods are based on weakening the restrictions on the solution
space in the continuous setting, and searching for the approximate solution in the
subspace which spans basis functions supported on small regions inside the domain.
These methods are well-suited to solving problems on complex domains, and are
therefore widely used in practical applications.
In this work we consider only finite element methods (FEMs) for solving partial
differential equations.
This chapter comprises an overview of several theoretical and practical aspects of classical FEMs. The theory and notation presented here are essential in developing
the techniques that form the core of this thesis. Most of the work presented in this chapter is based on work already presented in \citet{arthursthesis,asnerthesis,bernabeusthesis,burman2007unified}.
%
%Chris
%The method that we use for spatially discretising our equations in this work is the finite
%element method (FEM) for obtaining approximations to the solution of partial differential
%equations. It was introduced in its present form in 1942 by Richard L. Courant as an
%appendix to his paper on variational methods [75], although it does not then appear in
%the literature again for another sixteen years [76]. It was further developed by engineers
%during the late 1950s and early 1960s as a method of solving equations in structural engineering
%[77], and has since developed into a general tool for the numerical approximation
%of solutions to PDEs.