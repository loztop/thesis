{
\Large
\noindent\makebox[0in][l]{\xauthor}\hfill\makebox[6in][r]{Doctor of
Philosophy} \vskip 2pt
\noindent{\xcollege}\hfill\makebox[4.9in][r]{\xterm}
}

\vskip 1cm

{
\Large \bf
%\Large
\begin{center}
%{ \scshape A Computational Investigation of Acute Ischaemia-induced Changes in Cardiac Electrophysiology: \\  \large From Rabbit Experiments to Drug Safety in Human }
{\xtitle}
\end{center}
}

{
\large\bf
\begin{center}
Abstract
\end{center}
}

%\setlength{\baselineskip}{16truept}

%A stabilized nonlinear finite element method for the three-field (displacement, fluid flux and pressure) nonlinear quasi-static incompressible poroelasticity problem valid in large deformations is presented. We use the lowest possible approximation order: piecewise constant approximation for the pressure, and piecewise linear continuous elements for the displacements and fluid flux. Due to the discontinuous pressure approximation, sharp pressure gradients due to changes in material coefficients or boundary layer solutions can be captured reliably. For modelling purposes, we first derive the general model of poroelasticity, to highlight any modelling assumptions required to arrive at the standard quasi-static incompressible model of poroelasticity. We then present the linearization and discretisation of the equations, and give a detailed account of the implementation. 
%To remove spurious pressure oscillations, associated with the saddle point structure in the coupled equations, we introduce a stabilization term. 
%Numerical experiments in 3D verify the method and illustrate its ability to reliably capture steep pressure gradients.



