\section{Discussion}
\label{sec:conclusion}
We have presented a mathematical model of the lung that tightly couples tissue deformation with ventilation using a poroelastic model coupled to a fluid network model. We have highlighted the assumptions necessary to arrive at such a model, and outlined its limitations. In comparison with previous ventilation models, the current approach models the tissue as a continuum and is therefore able to regionally conserve mass (which means conserve volume as the solid skeleton and fluid are both incompressible), and to model collateral ventilation. Further it is driven by deformation boundary conditions extracted from imaging data to avoid having to prescribe a pleural pressure which is impractical to be measured experimentally. In simulations of normal breathing, the model is able to produce physiologically realistic global measurements and dynamics. In simulations with altered airway resistance and tissue stiffness, the model illustrates the interdependence of the tissue and airway mechanics and thus the importance of a fully coupled model.
%
\subsection{Contributors of airway resistance and tissue mechanics to lung function}
We have found that there is a strong correlation between airway resistance and ventilation, see Figure \ref{fig:ventilation_scatter}. Also, due to heterogeneity in airway resistance, hysteresis effects appear during breathing (Figure \ref{fig:recoil_trachea}) and result in a complex ventilation distribution, caused by delayed filling and emptying of the tissue. Due to the Poiseuille law that governs the flow through the airways, small changes in airway radii can result in large changes in pathway resistance, which in turn can significantly affect the results of the coupled model. Thus, parametrizing the airways correctly is very important. However this is notoriously difficult since CT data is only available down to the 5-6th generation, and small errors and biases in the segmentation, that get propagated by the airway generation algorithm, can have large influences in determining the simulation results. Changes in tissue elasticity coefficients also play an important role in determining the function of the lung model. This has been demonstrated in section \ref{sec:weakening} where are a reduction in the Young's modulus within a specified region causes significant changes in ventilation, pressure and stress.

%It is difficult to directly compare the importance of airway resistance and tissue mechanics since little is known about the distributions of these in the healthy or diseased lung.



%
\subsection{Limitations and future work}
\label{sec:future_work}
%
In order to move towards a more realistic model of the lung breathing, many steps need to be taken. We will list the main limitations that exist in the airway tree model, the poroelastic model, the boundary conditions and the geometry, and give indications on how these could be addressed in a future model.

\noindent \textbf{Airway tree limitations:} (1) The airway tree flow model currently implemented makes the Poiseuille flow assumption for the whole tree. The Poiseuille flow assumption requires flow to be fully developed and laminar. This may  be true for the smaller airways where the Reynolds number is small but is certainly false for the larger upper airways where high Reynolds number flows occur. Such a model will therefore not be able to capture the high Reynolds number flows and turbulent effects that are known to exists in the upper airways.
This could be improved by modifying the airway resistance at different generations according to the Reynolds number \cite{Swan2012,pedley1970energy}. Further improvements could be made by using a more sophisticated flow model for the airways, such as the 3D-0D model presented in \cite{ismail2013coupled}. (2) The coupling of each terminal branch to the tissue currently assumes that there is no added resistance to air flowing from the terminal branch to each alveolar unit within the tissue. This could be improved by adding a simple resistive (impedance) model considering the volume of tissue that the terminal branch is feeding. This would also slightly increase the mean pressure drop of the lung model. (3) At the moment the airway tree is assumed to be static, and its configuration is not influenced by the deformation and stresses in the tissue. This could be improved by modelling the interaction of stresses and strains on the airway wall, opening up the airways during inspiration.


\noindent \textbf{Poroelastic tissue limitations:} (1) We have assumed a Neo-Hookean law for strain energy law to make the interpretation of the elasticity constants and dynamics of the model as simple as possible. However lung parenchyma is known to follow an exponential stress-strain relation, especially past tidal volume, where a law such as the one proposed by \cite{fung1975stress} might be more appropriate. Also little is known about the form of the strain-energy law during disease (e.g. fibrosis or emphysema). Similarly, for the permeability law little is known about its form for healthy or diseased tissue. Further experiments and modelling investigation would be needed to develop these. (2) Currently the tissue has been parameterized homogeneously to simplify the analysis of the results. Density information from CT images could be used to parameterize the initial porosity and elasticity coefficients. (3) We have ignored the effect of blood in the tissue. The inertia and gravity forces of blood acting on the tissue could be of importance when
predicting deformation and ventilation in the lung. Due to the modular framework of the poroelastic theory it should be possible to include blood as a separate phase in a future version of the model. A vascular tree could also be generated from CT images and coupled to the poroelastic medium. (4) The airflow within the poroelastic tissue has been assumed to be inviscid. However, if we were to consider diseased states such as emphysema, where large areas of lung tissue completely break down leaving big holes, it could be argued that viscous forces could well play an important role, making it important to include them in our model. In a future version of the model the Darcy flow model could be replaced with a Brinkman, or even a Stokes flow model for big holes. %The proposed three-field poroelastic formulation makes it easy to do this.


\noindent \textbf{Boundary condition limitations:} (1) The current registration should be updated to a more sophisticated non-linear registration algorithm (e.g. \cite{jahani2014assessment,yin2013multiscale,heinrich2013mrf}) that is able to account for the complicated deformation of the lung surface during breathing. (2) It is known that the lung surface is able to slide freely within the plueral cavity. This feature could be implemented using methods already presented in \cite{kowalczyk1994modelling} and \cite{ateshian2010finite}.

\noindent \textbf{Geometry limitations:} (1) To model the complete organ and give a more accurate pressure drop, both the right and left lung, and the trachea and mouth should be included. (2) The airway tree generated in this work goes down to generations 8-13. More generations should be added to result in a fuller and more realistic tree. This would also require a finer mesh to approximate the lung tissue to resolve the coupling between each terminal branch and a subregion of lung tissue. (3) Cavities in the lung parenchyma due to large airways are currently not accounted for, i.e. it is assumed that the volume occupied by the airways is zero. To improve on this, a mesh of the lung with the larger upper airways removed would need to be generated. This new mesh could also incorporate a model of the cartilage found in the upper airways. (4) Additional no-flux boundaries should be introduced to represent the well defined and thought to be impermeable boundaries, between lobes (fissures) and lung segments.


%\noindent \textbf{Validation:} For this model to be of practical use it is crucial that it is properly validated. Computed tomography and 4D (dynamic) Magnetic resonance imaging (MRI) can be used to track displacements and calculate volume changes of lung structures. MRI of gases such as Hyperpolarized Xenon \cite{kaushikdiffusion} and Helium 3 can be used to infer the flow and diffusion of gases.


%\subsubsection{Modelling other tissues}
%\noindent \textbf{Other organs:}  The proposed methodology could be adapted to model other biological tissues where blood vessels flow through and interact with a deforming tissue. For example, when modelling perfusion of blood flow in the beating myocardium \cite{chapelle2010poroelastic,cookson2011novel}, modelling brain oedema \cite{li2010three} or hydrocephalus \cite{wirth2006axisymmetric}, or microcirculation of blood and interstitial fluid in the liver lobule \cite{leungchavaphongse2013mathematical}.



\section{Conclusion}
%
The model presented in this chapter can be used to investigate mechanical problems dependent on coupled deformation and ventilation in the lung.
%
%The model presented in this paper is a valid tool for solving the mechanical problem of tightly coupled lung deformation and ventilation during normal breathing and breathing with disease. 
The numerical simulations are shown to be able to reproduce global physiologically realistic measurements. A fully nonlinear formulation permits the inclusion of various constitutive models, allowing investigation into different diseased states during various breathing conditions. A finite element method has been used to discretize the equations in a monolithic way to ensure convergence of the nonlinear problem, even under strong poroelastic-fluid-network coupling conditions. Due to the flexibility of the model, further improvements in its physiological accuracy are possible. It is hoped that the model presented here can form the basis for studies on the importance of airway and tissue heterogeneity on lung function, testing of mechanical hypothesis for the progression of disease, and investigations into phenomena such as hyperinflation, fibrosis and constriction.



%I think you’re confusing validation with personalisation. I think the model is validated enough to perform mechanistic studies (and you should say this). You can’t use it in a patient specific setting without personalisation, which is a whole different game. If you want to go down that route read some of Jiahe Xi, Pablo Lamata and Nic Smith’s cardiac work.