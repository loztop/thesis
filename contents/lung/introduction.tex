\section{Introduction}
The aim of this chapter is not to present the most complete or accurate ventilation or deformation lung model to date. Instead we aim to present a new methodology and highlight some of the modelling assumptions required for a poroelastic lung model. We hope that this model will in future be extended to include sophisticated flow models of the airways, more advanced constitutive laws that make use of additional imaging data to parametrize the model, and improved registration algorithms, to yield a more realistic and accurate full organ lung model.

The rest of this chapter is organized as follows. In section \ref{sec:model} we present the assumptions and define the mathematical lung model and describe its implementation. In section \ref{sec:model_generation} we describe the generation of the computational lung geometry and boundary conditions, and in section \ref{sec:numerical_results} we present numerical simulations of tidal breathing, and investigate the effect of airway constriction and tissue weakening. Finally in section \ref{sec:conclusion}, we conclude and outline future work to improve the current lung model.
