\section{Model generation}
\label{sec:model_generation}

\subsection{Mesh generation}
We derive a whole organ lung model, of the right lung, from a high-resolution CT image taken at total lung capacity (TLC) and functional residual capacity (FRC). The bulk lung is first segmented from the CT data (slice thickness and pixel size 0.73 mm) using the commercially available segmentation software Mimics\footnotemark[1]. We then use the open-source image processing toolbox iso2mesh \cite{fang2009tetrahedral} to generate a Tetrahedral mesh containing 38369 elements. The conducting airways are also segmented from the CT data taken at TLC level, and a centerline with radial information is calculated. To approximate the remaining airways up to generation 8-13 we use a volume filling airway generation algorithm to generate a mesh of the airway tree containing 13696 nodes, with 2140 terminal branches \cite{bordas2014}.
%

\footnotetext[1]{http://biomedical.materialise.com/mimics}
\subsection{Reference state, boundary conditions and initial conditions}
\label{sec:ref_state_bcs}
%A reference state is typically chosen to associate with a stress-free state. 
%
The poroelastic framework we have described requires a stress free reference state. Biological tissues do not possess a `reference state' in space where the material is free of both stress and strain. The cells that make up tissues are born into stressed states and live out their lives in these stressed states \cite{freed2013implicit}.

In this work, we scale the lung from FRC to a configuration, the reference state, in which the internal stresses and strains are assumed to be zero. The geometry of the reference state is then used as the initial configuration
of the lung model. The lung model is then uniformly inflated from the reference state to create a pre-stressed FRC configuration which has a mean elastic recoil of about $0.49 \times 10^{3} \;\mbox{Pa} $, commonly understood to be a typical value \cite{west2008respiratory}. From there we simulate tidal breathing. A similar approach has also been used in \cite{lee1983finite}.

We register the expiratory (FRC) segmentation to the segmentation at TLC using a simple registration procedure that uses three scalings in the $x,y$ and $z$ direction to map between the bounding boxes of the segmentations at FRC and TLC. This yields a rough estimate of the deformation field for the lung surface from expiration to inspiration. To simulate tidal breathing we assume a sinusoidal breathing cycle and expand the lung surface from FRC to 40$\%$ of the deformation from FRC to TLC,
 \begin{equation}
 \boldsymbol{u}_{D}(t)=  0.2\left(1+\sin\left( \frac{t\pi}{2}+\frac{3\pi}{2}\right)\right)\boldsymbol{u}_{D,TLC}\;\;\; \mbox{on}\; \Gamma_{d}.
 \label{eqn:deform_bc}
 \end{equation}
Here $\boldsymbol{u}_{D,TLC}$ is the deformation of the lung surface from FRC to TLC, obtained using the registration procedure. For our mesh and registration this results in a physiologically realistic tidal volume of $0.59$ liters. We simulate breathing for a total of 8 seconds (2 breathing cycles) resulting in a breathing frequency of 15 breaths per minute. Due to the incompressibility of the poroelastic tissue this also determines the total volume of air inspired/expired and the flowrate at the trachea, see Figure \ref{fig:volume_trachea} and \ref{fig:flowrate_trachea} respectively.
%In future this method should be upgraded to include a non-linear registration procedure to give a more accurate description of the lung surface deformation.
For the fluid boundary condition we have that the whole lung is sealed so that no fluid can escape through the lung surface, with $\boldsymbol{z}\cdot \boldsymbol{n} =0$ along the whole boundary. For the airway network boundary condition we set the outlet pressure of the airway network to zero atmospheric pressure, $P_{0}=0$.
%
\subsection{Simulation parameters}
\label{sec:sim_params}
Several parameters for lung tissue elasticity and poroelasticity have been proposed \cite{zhang2004technical,werner2009patient,lande2006analysis,owen2001mechanics,de1981model}. There is no consensus in the values in the literature. In this study we have chosen parameters from the literature, as shown in Table \ref{tab:lung_sim_parameters}. These parameters are within range of existing models, and result in physiologically realistic simulation results (see section \ref{sec:numerical_results}).
%
\begin{table}[H]
\begin{center}
\scalebox{0.7}{
\begin{tabular}{ l c c c c}
\hline Parameter  & Value& Reference  \\
\hline
$\phi_{0}$ & $0.99$ & \cite{lande2006analysis} \\
${\kappa}_{0} $  & $10^{-5} \; \mbox{m}^{3}\,\mbox{s}\,\mbox{kg}^{-1}$ & \cite{lande2006analysis}\\
 $E $ & $0.73 \times 10^{3} \;\mbox{Pa}$ & \cite{de1981model} \\
 $\nu $ & $0.3$ & \cite{de1981model} \\
  $\mu_{f} $ & $1.92 \times 10^{-5} \; \mbox{kg}\,\mbox{m}^{-1}\,\mbox{s}^{-1}$ & \cite{Swan2012} \\
  \hline
   $T $ & $8s$ & - \\
    $\Delta t  $ & $0.2s$ & - \\
    $\delta  $ & $10^{-5}$ & - \\

\hline
\end{tabular}
}
\end{center}
\caption{Parameters for breathing simulations.}
\label{tab:lung_sim_parameters}
\end{table}
%Also note that we have chosen to not include gravity in our model. This is because we will not be comparing against imaging data and do not wish to additionally complicate the interpretation of simulation results. 