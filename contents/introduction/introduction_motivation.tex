%Airway diseases affect over 400 million people world-wide and cause considerable morbidity and mortality. Current therapies are inadequate and we do not have sufficient tools to predict disease progression or response to current or future therapies. Building computational models of the lungs will bridge the critical gaps in our clinical management of airways disease by providing models to predict disease progression and response to treatment. 

The main function of the lungs is to exchange gas between air and blood, supplying oxygen during inspiration and removing carbon dioxide by subsequent expiration. Gas exchange is optimised by ensuring efficient matching between ventilation and blood flow, the distributions of which are largely governed by tissue deformation, gravity and branching structure of the airway and vascular trees. In this work, we focus on the link between tissue deformation and ventilation. Understanding the interdependence between structure, and mechanical function in the lung has traditionally relied on direct measurement or medical imaging. Limitations of these approaches include difficulty in isolating a specific subsystem from its influence on the rest of the organ to gain an indepth understanding of the underlying mechanics. A carefully constructed computational model provides the advantage of exact control over functional parameters and the geometry of the solution domain, allowing for investigations into complex functional mechanisms. The work developed in this thesis is part of a longer term aim to link detailed anatomic imaging to computational analysis of structure-function relationships in the integrated pulmonary system through computational modelling of the lung and airway tree \citep{tawhai2006imaging}. 

Previous work has typically focused on modelling either ventilation or tissue deformation in isolation. However evaluation of each component (i.e. tissue deformation and ventilation) separately does not necessarily give accurate ventilation predictions or provide a good indication of how the integrated organ works, this is because both components are interdependent. To gain a better understanding of the biomechanics in the lung it is therefore necessary to fully couple the tissue deformation with the ventilation. To achieve this tight coupling between the tissue deformation and the ventilation we propose a multiscale model that approximates the lung parenchyma by a biphasic (tissue and air, ignoring blood) poroelastic model, that is then coupled to an airway fluid network model. 

An integrated model of ventilation and tissue mechanics will be particularly important for understanding respiratory diseases since nearly all pulmonary diseases lead to some abnormality of lung tissue mechanics \citep{suki2011lung}. For example, chronic obstructive pulmonary disease (COPD) encompasses emphysema (destruction of alveolar tissue) and chronic bronchitis which can cause severe, airway remodelling, bronchoconstriction and air trapping, all of which can significantly alter tissue properties. If the tissue mechanics are affected so too will the ventilation and vice versa, again emphasising the importance of a model that fully couples the ventilation and tissue mechanics in the lung. The impact of alterations during disease, such as airway narrowing or changes in tissue properties, on regional ventilation and tissue stresses are not well understood. For example, one hypothesis is that airway disease may precede emphysema \citep{galban2012computed}. The computational lung model could be applied to investigate the impact of airway narrowing and tissue stiffness during obstructive lung diseases on tissue stresses, alveoli pressure and ventilation.

Developing such a fully coupled model has to our knowledge not yet been achieved. There are many difficulties involved in creating a model that is physiologically accurate and can be solved numerically. We will need to develop methodology to solve the poroelastic equations, and develop solution techniques to couple the poroelastic model to the airway fluid network model.

In particular, in the diseased lung, abrupt changes in tissue properties and heterogeneous airway narrowing are possible. This can result in a patchy ventilation and pressure distribution \citep{venegas2005self}. In this situation existing methods that solve the poroelastic equations using a continuous pressure approximation would struggle to capture the steep gradients in pressure, and result in localized oscillations in the pressure. By developing a method that utilises a discontinuous approximation for the pressure we will be able to approximate these steep pressure gradients reliably, and avoid localized oscillations in the pressure.% \citep{white2008stabilized}.  

The proposed methodology could also be adapted to model other biological tissues where blood vessels flow through and interact with a deforming tissue. For example, when modelling perfusion of blood flow in the beating myocardium \citep{chapelle2010poroelastic,cookson2011novel}, modelling brain oedema \citep{li2010three} or hydrocephalus \citep{wirth2006axisymmetric}, or microcirculation of blood and interstitial fluid in the liver lobule \citep{leungchavaphongse2013mathematical}. In addition to this poroelasticity theory has also been used in various geomechanical applications ranging from reservoir engineering \citep{phillips2007coupling} to modelling earthquake fault zones \citep{white2008stabilized}. The theory developed in this thesis could be applied in these fields.


%Biot's poroelastic theory has been used in various geomechanical applications ranging from reservoir engineering \citep{phillips2007coupling} to modelling earthquake fault zones \citep{white2008stabilized}. More recently, fully saturated incompressible poroelastic models, often derived using the theory of mixtures \citep{boer2005trends}, have been used for modelling biological tissues. For example, modelling lung parenchyma \citep{kowalczyk1993mechanical}, protein based hydrogels embedded within cells \citep{galie2011linear}, perfusion of blood flow in the beating myocardium \citep{chapelle2010poroelastic,cookson2011novel}, the modelling of brain oedema \citep{li2010three} and hydrocephalus \citep{wirth2006axisymmetric}, along with the modelling of interstitial fluid and tissue in articular cartilage and intervertebral discs \citep{mow1980biphasic,holmes1990nonlinear,galbusera2011comparison}.