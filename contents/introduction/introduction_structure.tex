The contributions of each chapter to the thesis are as follows:\newline

\noindent \textbf{Chapter 2:} We give a brief overview of lung physiology, review the literature on ventilation models and existing porelastic models, and discuss numerical methods currently available to solve the poroelastic equations.\newline

\noindent \textbf{Chapter 3:} We introduce the general theory of poroelasticity valid in large deformations, and derive the linear poroelastic equations, valid in small deformations.\newline

\noindent \textbf{Chapter 4:} We
outline the basic concepts of the standard
continuous Galerkin finite element method. We then discuss mixed problems and their stability requirement.
\newline

\noindent \textbf{Chapter 5:} A stabilized conforming finite element method for the linear three-field (displacement, fluid flux and pressure) poroelasticity problem is presented. By applying a local pressure jump stabilization term to the mass conservation equation we avoid pressure oscillations. For the fully discretized problem we prove existence and uniqueness, an energy estimate and an optimal a-priori error estimate. Numerical experiments in 2D and 3D illustrate the convergence of the method, show the effectiveness of the method to overcome spurious pressure oscillations, and evaluate the added mass effect of the stabilization term.
\newline
%
%in section \ref{sec:themodel}, we describe the model equations; in section \ref{sec:weak_formulation} we present the continuous weak formulation of the model; in section \ref{sec:fully_discrete} we introduce the fully-discrete model, prove existence and uniqueness at each time step, and give an energy estimate over time. We then derive an optimal order a-priori error estimate in section \ref{sec:error}. Finally in section \ref{sec:numerical_simulations}, we present some numerical experiments to illustrate our theoretical findings in 2D and 3D, test the robustness of the method, and demonstrate its ability to overcome pressure oscillations.

\noindent \textbf{Chapter 6:} We apply the method developed in Chapter 5 to solve the three-field nonlinear quasi-static incompressible poroelasticity problem valid in large deformations. We present the linearization and discretisation of the equations, and give a detailed account of the implementation. Numerical experiments in 3D verify the method and illustrate its ability to reliably capture steep pressure gradients.
\newline

\noindent \textbf{Chapter 7:} We present the model assumptions required for the proposed poroelastic lung model and outline its mathematical formulation and coupling to the airway fluid newtork. A finite element method is presented to discretize the equations in a monolithic way to ensure convergence of the nonlinear problem. Finally, numerical simulations on a realistic lung geometry that illustrate the coupling between the poroelastic medium and the network flow model are presented. Numerical simulations of tidal breathing are shown to reproduce global physiological realistic measurements. We also investigate the effect of airway constriction and tissue weakening on the ventilation, stress and alveolar pressure distribution.
\newline

\noindent \textbf{Chapter 8:} We summarise the main results and propose future lines of research.
\newline



 
 


