The main goal of this thesis is to rigorously develop a finite element method for solving the poroelastic equations, and then use this methodology to simulate the lung breathing on a realistic geometry. More specific targets are:
\begin{enumerate}%[label=\roman*]
 \item Develop a practical low-order finite element method for solving the linear poroelastic equations using a discontinuous pressure approximation. Prove theoretical results about the discretisation, including existence and uniqueness, an energy estimate and an optimal a-priori error estimate.
\item Extend the method to a non-linear finite element method to solve the poroelastic equations valid in large deformations.
 \item Rigorously test the method using numerous test problems to verify theoretical stability and convergence results, and its ability to reliably capture steep pressure gradients. 
 \item Derive a poroelastic model for lung parenchyma coupled to an airway fluid network model, and develop a stable method to numerically solve the coupled model.
  \item Solve the computational lung model on a realistic geometry, with boundary conditions extracted from imaging data, to simulate breathing. Evaluate the effect of tissue weakening and airway narrowing on lung function.
%\item Solve the comptaional lung model on a realistic geomerty and physiological realistic boundary conditions to simulate breathing.
\end{enumerate}



%
%
%In this thesis, we investigate the potential of - lung model see if it can couple deformation and ventilation.
%
%see if we can find a method, test  it etc.
%
%Numerics:
%Solid Theory, small and large + nuermical tests.
%
%Lung model:
%The aim of this work is not to present the most complete or accurate ventilation or deformation lung model to date. Instead we aim to present a new methodology and highlight some of the modelling assumptions required for a poroelastic lung model. We hope that this model will in future be extended to include sophisticated flow models of the airways, more advanced constitutive laws that make use of additional imaging data to parametrize the model, and improved registration algorithms, to yield a more realistic and accurate full organ lung model.