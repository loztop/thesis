Two complementary approaches have been developed for modelling a deformable porous medium. Mixture theory, also known as the Theory of Porous Media (TPM) \cite{bowenporouslectures,bowen1980incompressible,boer2005trends}, has its roots in the classical theories of gas mixtures and makes use of a volume fraction concept in which the porous medium is represented by spatially superposed interacting media. An alternative, purely macroscopic approach is mainly associated with the work of Biot, a detailed description can be found in the book by Coussy \cite{coussy2004poromechanics}. Relationships between the two theories are explored by \cite{coussy1998mixture}. As is most common in biological applications, we use the mixture theory for poroelasticity as outlined in \cite{boer2005trends}.