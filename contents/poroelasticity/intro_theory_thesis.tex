We will now develop the theory required for modelling of the complete porous medium
made up of a solid and fluid phase. There exist two main approaches for modelling a deformable
porous medium. There is the mixture theory approach also known as the Theory of Porous Media (TPM)
\citep{bowenporouslectures,bowen1980incompressible,boer2005trends}, which has its roots in the classical theories of gas mixtures, and makes use
of a volume fraction concept, where the porous medium is represented by spatially superposed interacting media. The other approach is purely
macroscopic and is mainly associated with the work of Biot, a detailed description can be found
in the book by Coussy \citet{coussy2004poromechanics}. For a good comparison between the two theories see \citet{coussy1998mixture}. In this work we will use
the mixture theory approach, outlined in the book by R. Boer \citet{boer2005trends}, to derive the equations since this is the common approach taken
for applications of poroelasticity in biology.

