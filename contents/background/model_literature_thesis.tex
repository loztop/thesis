\section{Computaional lung models}
\label{section:review_models}
%
There exist a large number of computational ventilation and deformation models for the lung. Some models are designed to model particular phenomena whilst others are more general. They also range in spatial complexity from 0D compartment type models to 3D models which are able to incorporate `patient-specific' geometries extracted from CT images. In this review, we will focus on models that couple ventilation with tissue deformation and can be used as patient-specific models. A review of popular lung models can be found in \cite{bates2009lung}. 
%
One study that couples ventilation and tissue deformation using a one way coupling approach and then applies the model to a full 3D geometry is described in \cite{tawhai2010image}. Here a mechanics model for elastic deformation of compressible lung tissue is used to provide flow and pressure boundary conditions for an embedded airway model which makes the resultant ventilation distribution dependent on the tissue deformation due to gravity. In \cite{Swan2012}, air flow is simulated in patient specific conducting airways which are coupled to geometrically simplified terminal acinar units with varying volume dependent compliances. The fluid flow in the airways is approximated by Poiseuille flow with an added correction term for airway bifurcations. The end terminal acinar units are able to expand but are assumed to be independent of neighbouring acinar units. This does not allow for feedback from neighbouring acini that are infact tightly connected by a matrix of fibers, collagen and capillaries. 
%This model confirms experimental evidence that in the healthy lungs tissue compliance has a far greater effect than airway resistance on the spatial distribution of ventilation, and hence a realistic description of tissue deformation is essential in models of ventilation. 
Other sophisticated flow models of the whole airway tree, as previously mentioned, also exist \citep{yin2013multiscale,ismail2013coupled}. These models solve the full 3D Navier-Stokes equations in the upper airways, segmented from CT images, to capture high Reynolds number effects. The 3D fluid model is then coupled to a 0D laminar flow model of the lower airways. %None of these models incorporate the feedback of tissue deformation on ventilation and vice versa, and only loosely couple tissue deformation to ventilation.  


\section{Poroelastic models for lung parenchyma and other biological tissue}
%\label{section:poroelastic_review}
Some early work on a mechanical model of lung parenchyma as a poroelastic medium has already been proposed in \cite{kowalczyk1993mechanical}. This work developed a similar poroelastic model to the one we propose, however it has only been applied to a very simple 2D geometry. Also in \cite{owen2001mechanics} homogenisation theory has been used to derive macroscopic poroelastic equations for average air flows and tissue displacements in lung parenchyma during high frequency ventilation. The resulting model is a one dimensional system of equations that is used to investigate the effect of high-frequency ventilation on strain in the parenchymal tissue. The use of a poroelastic model has also been applied to modelling other biological tissues. For example modelling protein based hydrogels embedded with cells \cite{galie2011linear}, perfusion of blood flow in the beating myocardium \cite{chapelle2010poroelastic}, \cite{cookson2011novel}, the modelling of brain oedema (swelling) \cite{li2010three} and hydrocephalus \cite{wirth2006axisymmetric}. Another application is the modelling of interstitial fluid and tissue in articular cartilage and intervertebral discs \cite{mow1980biphasic}, \cite{holmes1990nonlinear},\cite{galbusera2011comparison}.




\begin{comment}
%One study presented a framework for coupling models of ventilation and tissue deformation \cite{tawhai2006imaging}. First, changes in geometry of the lung mesh are determined and used to compute flow through an airway model. Pressures are then interpolated throughout the lung mesh to provide hydrostatic pressures that act as input loads for the mechanics model, which is then solved to predict shape change. These deformations are used to compute local volume changes which feed back as inputs to the flow-pressure problem and the system is solved iteratively. In these models tissue deformation is not directly dependent on ventilation, however an integrated model of ventilation and tissue mechanics model will be particularly important for understanding respiratory diseases since nearly all pulmonary diseases lead to some abnormality of lung tissue mechanics  \cite{suki2011lung}. For example, Chronic Obstructive Pulmonary Disease (COPD) encompasses emphysema (destruction of alveolar tissue) and chronic 
bronchitis which can cause severe bronchoconstriction and atelectasis (air trapping), both of which can significantly alter tissue properties. If the tissue mechanics are affected so too will the ventilation and vice versa emphasising the importance of a model that fully couples the tissue mechanics and ventilation in the lung. 

%There has been one study that presents a framework for coupling models of ventilation and tissue deformation \cite{tawhai2006imaging}. First, changes in geometry of the lung mesh are determined and used to compute flow through an airway model. Pressures are then interpolated throughout the lung mesh to provide hydrostatic pressures that act as input loads for the mechanics model, which is then solved to predict shape change. These deformations are used to compute local volume changes which feed back as inputs to the flow-pressure problem and the system is solved iteratively.

 %Although it is assumed that lung tissue is a continuum with uniform material properties, simulations of tissue deformation in a realistic geometry can give rise to a considerable degree of heterogeneity. 
%Including this model of tissue deformation in a ventilation model clearly predicts more physiologically consistent ventilation distributions than simply assuming that tissue compliance is constant or proportional to lung height. Therefore we conclude  that it is an essential feature in functional computational models of ventilation which aim to describe ventilation and perfusion matching or changes in ventilation distribution with disease.
%Another model that incorporates subject-specific anatomical geometry and approximates the airways by Poiseuille flow with an added correction term for airway bifurcations is presented in \cite{HedgesThesis}, here the aim is to predict forced expiration (FEV1) values. 

\end{comment}



 
%The model confirms experimental evidence that in the healthy lungs tissue
%compliance has a far greater effect than airway resistance on the spatial distribution of ventilation, and hence a realistic description of tissue deformation is essential in models of ventilation.
%Although it is assumed here that lung tissue is a continuum with uniform material properties, simulations of tissue deformation in a curvilinear geometry can give rise to a considerable degree of heterogeneity. Including this model of tissue deformation in a ventilation model clearly predicts more physiologically consistent ventilation distributions than simply assuming that tissue compliance is constant or proportional to lung height. Therefore we conclude  that it is an essential feature in functional computational models of ventilation which aim to describe ventilation and perfusion matching or changes in ventilation distribution with disease.''  \cite{SwanThesis}

 %However in this model, tissue deformation is not directly dependent on ventilation.


%Due to the often high Reynolds numbers in the upper airways, 3D fluid flow models have been used to model this sections of the airway tree. Clearly approximating the entire airway tree using 3D fluid flow remains intractable due to constraints on imaging resolution and computational power. Therefore smaller airways still need to be generated using a volume-filling branching algorithm \cite{tawhai2004ct} and then approximated by simpler 1D flow equations. The coupling of a 3D flow model to a 1D flow model for a complete airway tree has already been done in \cite{lin2009multiscale} and is achieved through mass conservation. i.e. the flow rates at the exit faces of the 3-D airway model can be determined by summing the flow rates at the terminal bronchioles using the connectivity information between the 3-D airway exit faces and the associated downstream 1-D airway branches. However this summing of flow rates is unable to account for any changes in resistances in the airway tree which might have been introduced 
%due to disease, for example bronchoconstriction or the collapsing of airways which is common in many types of COPD. There has also been sophisticated work on 3D to 1D fluid flow coupling for modelling blood flow with compliant vessels in the cardiovascular system \cite{formaggia2001coupling} and cerebral vasculature \cite{passerini20093d}.
