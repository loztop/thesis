\section{Finite element methods for poroelasticity}
The method that we use for spatially discretising our equations in this work is the finite element method (FEM) for obtaining approximations to the solution of partial differential equations.

After many decades of research there remain numerous challenges associated with the numerical solution of the poroelastic equations. When using the finite element method the main challenge is to ensure stability and convergence of the method and prevent numerical instabilities that often manifest themselves in the form of spurious oscillations in the pressure. It has been suggested that this problem is caused by the saddle point structure in the coupled equations resulting in a violation of the famous Ladyzhenskaya-Babuska-Brezzi (LBB) condition \citep{haga2012causes}, highlighting the need for a stable combination of mixed finite elements. Another numerical challenge in practical 3D applications is the algebraic system arising from the finite element discretisation. This can lead to a very large matrix system that has many unknowns and is severely ill-conditioned, making it difficult to solve using standard iterative solvers. Therefore low-order finite element methods that allow for efficient preconditioning are preferred \citep{white2011block,ferronato2010fully}.

\subsection{Linear three-field discretisations}
The poroelastic equations are often solved in a reduced displacement and pressure formulation, from which the fluid flux can then be recovered \citep{murad1994stability,white2008stabilized}. \citet{murad1994stability} have analysed the stability and convergence of this reduced displacement pressure $(\boldsymbol{u}/ p)$ formulation and were able to show error bounds for inf-sup stable combinations of finite element spaces (e.g. Taylor-Hood elements). In this paper we will keep the fluid flux variable resulting in a three-field, displacement, fluid flux, and pressure formulation. Keeping the fluid flux as a primary variable has the following advantages:
\begin{enumerate}[label=\roman*]
 \item It avoids the calculation of the fluid flux in post-processing. 
\item Physically meaningful boundary conditions can be applied at the interface when modelling the interaction between a fluid and a poroelastic structure \citep{badia2009coupling}.
 \item It allows for greater accuracy in the fluid velocity field. This can be of interest whenever a consolidation model is coupled with an advection diffusion equation, e.g. to account for thermal effects, contaminant transport or the transport of nutrients or drugs within a porous tissue \citep{khaled2003role}.
 \item It allows for an easy extension of the fluid model from a Darcy to a Brinkman flow model, for which there are numerous applications in modelling biological tissues \citep{khaled2003role}.
 \item It reduces the order of the spatial derivative of the pressure, allowing for a discontinuous pressure approximation without any additional penalty terms.
\end{enumerate}

\citet{phillips2007coupling,phillips2007couplingtwo}, have proven error estimates when solving the three-field formulation problem using continuous piecewise linear approximations for displacements and mixed low-order Raviart Thomas elements for the fluid flux and pressure variables. However this method was found to be susceptible to spurious pressure oscillations \citep[see,][]{phillips2009overcoming}. To overcome these pressure oscillations, \citet{li2012discontinuous} analysed a discontinuous three-field method, and \citet{yi2013coupling} analysed a nonconforming three-field method.
%In \citet{li2012discontinuous}, a discontinuous method and in \citet{yi2013coupling} a nonconforming three-field method is analysed. These papers were motivated by the need for a method that is able to overcome pressure oscillations \citep[see,][]{phillips2009overcoming} experienced by the method in \citet{phillips2007coupling,phillips2007couplingtwo}. 
In addition to these monolithic approaches there has been considerable work on operating splitting (iterative) approaches for solving the poroelastic equations \citep{wheeler2007iteratively,feng2010fully,kim2011stability}. Although these methods are often able to take advantage of existing elasticity and fluid finite element software, and result in solving a smaller system of equations, these schemes are often only conditionally stable. To ensure that the method is unconditionally stable, monolithic approaches are often preferred. The method proposed in this work is monolithic, and will therefore retain the advantage of being unconditionally stable .

%\subsection{Stability of Darcy-Stokes formulation}
%Numerous stabilization techniques for finite element methods have already been proposed in order to satisfy the LBB condition, most extensively for the model equations of Stokes and Darcy flow, which despite their simplicity retain all the difficulties of a saddle point problem. This will be discussed again in more detail in section \ref{sec:mixed}. Another good introduction on stabilization techniques for the Stokes problem can be found in chapter 5 of \citet{elman2005finite}. For a comparison of low-order stabilization techniques for the Darcy problem we refer to \citet{bochev2006computational}. Most stabilized methods lead to a modified variational formulation in which an additional term is added to the mass balance equation, modifying the incompressibility constraint
%in such a way that stability of the mixed formulation is increased,
%while still maintaining optimal convergence of the method. These stabilization techniques are of great interest to us since solving the three-field poroelasticity problem is essentially equivalent to coupling the Stokes equations (elasticity of the porous mixture) with the Darcy equations (fluid flow through pores), with a modified incompressibility constraint that combines the divergence of the displacement velocity and the fluid flux. 







%After many decades of research there remain numerous challenges associated with the numerical solution of the poroelastic equations valid in small and large deformations. When using the finite element method the main challenge is to ensure stability and convergence of the method and prevent numerical instabilities that often manifest themselves in the form of oscillations in the pressure. It has been suggested that this problem is caused by the saddle point structure in the coupled equations resulting in a violation of the famous Ladyzhenskaya-Babuska-Brezzi (LBB) condition \citep{haga2012causes}, highlighting the need for a stable combination of mixed finite elements. Another numerical challenge in practical 3D applications is the algebraic system arising from the finite element discretization. These theoretical stability issues have already been studied for some linear (small deformation) poroelasticity formulations \citep{murad1994stability,haga2012causes} and numerous remedies have been proposed \citep{phillips2007coupling,phillips2007couplingtwo,li2012discontinuous,yi2013coupling,berger2014stabilized}.
\subsection{Methods valid in large deformations}


We will now give a brief overview of different approaches for solving the poroelastic equations valid in large deformations. There has been some work on operating splitting (iterative) approaches where the poroelastic equations are separated into a fluid problem and deformation problem \citep{chapelle2010poroelastic}. Again, this approach is only conditionally stable. Some notable quasi-static incompressible large deformation monolithic approaches include a mixed-penalty formulation, and a mixed solid velocity-pressure formulation, both outlined in \citep{almeida1998finite}, the solid velocity-pressure formulation is similar to the commonly used reduced $(\boldsymbol{u}/p)$ formulation \citep{ateshian2010finite}. These two-field $(\boldsymbol{u}/p)$ formulations require a stable mixed element pair such as the popular Taylor-Hood element to satisfy the LBB inf-sup stability requirement. To reduce the number of unknowns, and allow for an equal-order, piecewise linear approximation, a stabilized reduced $(\boldsymbol{u}/ p)$ formulation has been proposed in \citep{white2008stabilized}. This method introduces a stabilization term to the mass conservation equation to overcome the spurious pressure oscillations. The key difficulty, however, that this stabilized element cannot escape is that jumps in material coefficients may introduce large solution gradients across the
interface, requiring severe mesh refinement. This is because a continuous pressure element is used, which is unable to reliably capture jumps in the pressure solution \citep{white2008stabilized}. In \citep{levenston1998variationally} a three-field (displacement, fluid flux, pressure) formulation has been outlined, however this method uses a low-order mixed finite element approximation without any stabilization and therefore is not inf-sup stable.





 