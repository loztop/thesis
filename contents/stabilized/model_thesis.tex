%%%%%%%%%%%%%%%%%%%%%%%%%%%%%%%%%%%%%%%%%%%%%%%%%%%%%%%%%%%%%%%%%%%%%%%%%%%%%%%%%%%%%%%%%%%%%%%%%%%%
%                                    The poroelastic model                                         %
%%%%%%%%%%%%%%%%%%%%%%%%%%%%%%%%%%%%%%%%%%%%%%%%%%%%%%%%%%%%%%%%%%%%%%%%%%%%%%%%%%%%%%%%%%%%%%%%%%%%

\section {The poroelastic model}
\label{sec:model}


%%%%%%%%%%%%%%%%%%%%%%%%%%%%%%%%%%%%%%%%%%%%%%%%%%%%%%%%%%%%%%%%%%%%%%%%%%%%%%%%%%%%%%%%%%%%%%%%%%%%
%     Governing equations
%%%%%%%%%%%%%%%%%%%%%%%%%%%%%%%%%%%%%%%%%%%%%%%%%%%%%%%%%%%%%%%%%%%%%%%%%%%%%%%%%%%%%%%%%%%%%%%%%%%%

\subsection{Governing equations}
\label{sec:nondimensional}

Following \cite{phillips2007coupling} and \cite{showalter2000diffusion}, we recall the governing equations (\ref{eqn:linear_simple_system}) for a fully saturated, incompressible poroelastic model
\begin{subequations}
\label{strong_cont_system}
\begin{align}
-(\lambda+\mu) \nabla (\nabla\cdot\mathbf{u}) - \mu \nabla^{2} \mathbf{u} +   \nabla p = \mathbf{f} \quad &\mbox{in}\; \Omega\times (0,T),\\
\perminv \mathbf{z} + \nabla p =  \mathbf{b} \quad   &\mbox{in}\; \Omega\times (0,T), \\
 \nabla \cdot ( \mathbf{u}_{t} + \mathbf{z} )  = g \quad  &\mbox{in}\; \Omega\times (0,T), \\
\mathbf{u} = \mathbf{u}_{D} \quad & \mbox{on}\; \Gamma_{D}\times (0,T), \\
\mathbf{\sigma}\mathbf{n} = \mathbf{t}_{N} \quad  &\mbox{on}\; \Gamma_{N}\times (0,T), \\
\mathbf{z} \cdot \mathbf{n} = {q_{D}} \quad  &\mbox{on}\; \Gamma_{F}\times (0,T), \\
p = p_{D}  \quad  &\mbox{on}\; \Gamma_{P}\times (0,T), \\
\mathbf{u}(0,\cdot) = \mathbf{u}^{0}\quad  &\mbox{in}\;\Omega.
\end{align}
\end{subequations}
%where $\dispcont$ is the displacement, $\fluxcont$ is the fluid flux and $\pcont$ is the pressure. Here $\mathbf{f}$ is the body force on the solid, $\mathbf{b}$ is the body force on the fluid and $g$ is the fluid source term, $\lambda$ and $\mu$ are the first and second Lam\'e parameters respectively, and $\perm$ is the dynamic permeability tensor. We will assume the Biot-Willis constant $\alpha=1$, and the constrained specific storage coefficient $c_0=0$. Our analysis can be extended to the situation when these last two assumptions do not hold, but it is necessarily more complicated. We consider $\Omega$ to be a bounded domain in $\mathbb{R}^{2}$ or $\mathbb{R}^{3}$, and for the purpose of defining boundary conditions, $\partial\Omega=\Gamma_D \cup \Gamma_N$ for displacement and stress boundary conditions and $\partial\Omega=\Gamma_P \cup \Gamma_F$ for pressure and flux boundary conditions, with outward pointing unit normal $\mathbf{n}$.
\begin{rem}
\label{remark:bdy}
Since the above resulting system of equations is linear, for ease of presentation,  we will assume all Dirichelet boundary conditions are homogeneous, ie., $\mathbf{u}_{D} = {\bf 0}, {q_{D}} = 0, p_{D}=0$.  %Hence, $\mixedspace = \mixedspacetest$.
\end{rem}



%%%%%%%%%%%%%%%%%%%%%%%%%%%%%%%%%%%%%%%%%%%%%%%%%%%%%%%%%%%%%%%%%%%%%%%%%%%%%%%%%%%%%%%%%%%%%%%%%%%%
%     Weak formulation
%%%%%%%%%%%%%%%%%%%%%%%%%%%%%%%%%%%%%%%%%%%%%%%%%%%%%%%%%%%%%%%%%%%%%%%%%%%%%%%%%%%%%%%%%%%%%%%%%%%%

\subsection{Weak formulation}
\label{sec:weak_formulation}

We define the following spaces for displacement, fluid flux and pressure respectively,
\begin{eqnarray*}
\dispspace &=& \lbrace \dispcont \in (H^{1} (\Omega))^d :\dispcont= {\bf 0} \;\mbox{on} \;\Gamma_{D} \rbrace, \\
\fluxspace &=& \lbrace \fluxcont \in H_{div}(\Omega):\fluxcont \cdot \normal = 0 \;\mbox{on} \;\Gamma_{F} \rbrace, \\
\pspace    &=& \left\lbrace
  \begin{array}{l l}
    L^{2}(\Omega)     &\; \text{if} \; \Gamma_{N} \cup \Gamma_{P} \neq \emptyset \\
    L^{2}_{0}(\Omega) &\; \text{if} \;\Gamma_{N} \cup \Gamma_{P}=\emptyset,
  \end{array} \right \rbrace,
\end{eqnarray*}
where
$L^{2}_{0}(\Omega) = \left\lbrace q \in L^{2}(\Omega) : \int_{\Omega} q\;\mbox{d}x=0\right\rbrace,$
which we combine to construct the mixed solution space
\begin{equation*}
\mixedspace = \left\lbrace \dispspace  \times \fluxspace  \times  \pspace \right\rbrace.
\end{equation*}
%The test functions lie in $\mixedspacetest = \left\lbrace \mathbf{W}^{E}_{0}  \times \mathbf{W}^{D}_{0}  \times  \mathcal{L}(\Omega)\right\rbrace$ where $\mathbf{W}^{E}_{0}=\lbrace \dispconttest \in (H^{1}( \Omega))^d:\dispconttest=0 \;\mbox{on} \;\Gamma_{D} \rbrace$, and $\mathbf{W}^{D}_{0}=\lbrace \fluxconttest \in H_{div}( \Omega):\fluxconttest \cdot \normal =0 \;\mbox{on} \;\Gamma_{F} \rbrace$.
We define the bilinear form
\begin{equation*}
a(\dispcont,\dispconttest) = \int_{\Omega}2\mu (\epsilon(\dispcont):\epsilon(\dispconttest))
                           + \lambda (\nabla \cdot \dispcont)(\nabla \cdot \dispconttest) \; {\rm d}x,
\end{equation*}
for $\dispcont, \dispconttest \in \dispspace$. This bilinear form is continuous such that
\begin{equation}
\label{eqn:a_cont}
a(\dispcont,\dispconttest) \leq C_{c} \honenorm{ \dispcont }\honenorm {\dispconttest}
  \; \forall \dispcont,\dispconttest \in (H^{1}(\Omega))^d.
\end{equation}
Using Korn's inequality \cite{brenner2008mathematical,ciarlet2002}, and $\int_{\Omega} \lambda (\nabla \cdot \dispconttest  )(\nabla \cdot \dispconttest  ) \geq 0 $ we have
\begin{equation}
\anorm{\dispconttest}^{2}=a( \dispconttest,\dispconttest)\geq 2\mu \ltwonorm{\epsilon(\dispconttest)}^{2}\geq C_{k}\honenorm{\dispdisctest}^{2}\;\;\forall \dispconttest \in \dispspace.
\label{eqn:a_coercive}
\end{equation}
Since $\perm$ is assumed to be a symmetric and strictly positive definite tensor, there exists eigenfunctions $\lambda_{min},\lambda_{max} > 0 $ such that $\forall \mathbf{x} \in \Omega, \; \lambda_{min} \ltwonorm{\mathcal{\mathbf{\eta}}} \leq  \mathbf{\eta}^{t} \perm(\mathbf{x})\mathbf{\eta} \leq  \lambda_{max} \ltwonorm{\mathcal{\mathbf{\eta}}} \;\; \forall \mathbf{\eta} \in \mathbb{R}^{d}$, and
\begin{equation}
\label{eqn:m_coercive}
{\lambda_{min}^{-1}}\ltwonorm{\fluxconttest}^{2} \geq (\perminv \fluxconttest,\fluxconttest) \geq {\lambda_{max}^{-1}}\ltwonorm{\fluxconttest}^{2} \; \forall \fluxconttest \in \fluxspace.
\end{equation}

The continuous weak problem is: Find $\dispcont(x,t) \in \dispspace$, $\fluxcont(x,t) \in \fluxspace$, and $\pcont(x,t) \in \pspace$ for any time $t\in[0,T]$ such that:
\begin{subequations}
\label{eqns:weak_cont_system}
\begin{align}
& a(\dispcont,\dispconttest) - (\pcont,\nabla \cdot \dispconttest)=(\mathbf{f},\dispconttest) + (\mathbf{t}_{N},\dispconttest)_{\Gamma_{N}} \; \forall \dispconttest\in \dispspace,
\label{eqn:weak_cont_elast_mom} \\
&(\perminv \fluxcont,\fluxconttest) - (\pcont,\nabla \cdot \fluxconttest) = (\mathbf{b},\fluxconttest)
%- (p_{D},\fluxconttest\cdot \normal)_{\Gamma_{P}}
\; \forall\fluxconttest\in \fluxspace,
\label{eqn:weak_cont_flux_mom} \\
&(\nabla \cdot  {\dispconttime},\pconttest) + (\nabla \cdot \fluxcont,\pconttest) = (g,\pconttest) \; \forall\pconttest\in \pspace. \label{eqn:weak_cont_mass}
\end{align}
\end{subequations}
We will assume the following regularity requirements on the data,
\begin{equation}
\label{eqns:weak_cont_system_data}
\begin{gathered}\begin{aligned}
&\mathbf{f}\in C^{1}([0,T]; (H^{-1}(\Omega))^{d}),  \\
&\mathbf{b}\in C^{1}([0,T]; H_{div}^{-1}(\Omega)),
\end{aligned}\end{gathered}
\qquad
\begin{gathered}\begin{aligned}
&\mathbf{t}_{N}\in C^{1}([0,T]; H^{-1/2}(\Gamma_{N})), \\
&g\in C^{0}([0,T]; (L^{2}(\Omega))^{d}).
\end{aligned}\end{gathered}
%&\dispcont_{D}\in C^{1}([0,T]; H^{1/2}(\Gamma_{D})),
%&p_{D}\in C^{0}([0,T]; L^{2}(\Gamma_{P})), \\
%&q_{D}\in C^{0}([0,T]; TrW), \\
\end{equation}
%where $TrW:=\left\lbrace \mathbf{w}\cdot\mathbf{n}|_{\Gamma_{f}}: \mathbf{w}\in H_{div}(\Omega)  \right\rbrace$.
For the initial conditions we require that $\mathbf{u}^{0} \in  (H^{1}(\Omega))^{d}$.
%\begin{equation}
%\mathbf{u}^{0} \in  (H^{1}(\Omega))^{d}.%,\;\;\; \mathbf{z}^{0} \in  (H_{div}(\Omega))^{d}, \;\;\; p^{0} \in \pspace.
%\label{eqn:initial}
%\end{equation}
{The well-posedness of the continuous two-field formulation has been proven by \cite{showalter2000diffusion}. \cite{lipnikov2002numerical} proves well-posedness for the continuous three-field formulation (\ref{eqns:weak_cont_system}). In this work we also establish the well-posedness of (\ref{eqns:weak_cont_system}) as a result of the energy estimates proven in section \ref{sec:energy}, see remark \ref{remark:wellposedness}.}

 

%%%%%%%%%%%%%%%%%%%%%%%%%%%%%%%%%%%%%%%%%%%%%%%%%%%%%%%%%%%%%%%%%%%%%%%%%%%%%%%%%%%%%%%%%%%%%%%%%%%%
%     Fully-discrete model
%%%%%%%%%%%%%%%%%%%%%%%%%%%%%%%%%%%%%%%%%%%%%%%%%%%%%%%%%%%%%%%%%%%%%%%%%%%%%%%%%%%%%%%%%%%%%%%%%%%%

\subsection{Fully-discrete model}
\label{sec:fully_discrete}
%Let $\mathcal{T}^{h}$ be a partition of $\Omega$ into non-overlapping elements $K$, where $h$ denotes the size of the largest element in $\mathcal{T}^{h}$ and assume that the partition is quasi-uniform. 
We define the following finite element spaces,
\begin{eqnarray*}
\dispspacedisc &=& \left\lbrace \dispdisc  \in C^{0}(\Omega): \dispdisc |_{K} \in P_{1}(K) \ \forall K \in \mathcal{T}^{h}, \dispdisc  = 0 \; \mbox{on} \;\Gamma_{D} \right\rbrace,  \\
\fluxspacedisc &=&\left\lbrace \fluxdisc  \in C^{0}(\Omega): \fluxdisc |_{K} \in P_{1}(K) \ \forall K \in \mathcal{T}^{h},  \fluxdisc \cdot \normal = 0 \; \mbox{on} \;\Gamma_{F} \right\rbrace, \\
\pspacedisc &=& \left\lbrace
  \begin{array}{l l}
    \left\lbrace \pdisc: \pdisc |_{K} \in P_{0}(K) \ \forall K \in \mathcal{T}^{h} \right\rbrace & \; \text{if $\Gamma_{N} \cup \Gamma_{P} \neq \emptyset$}\\
    \left\lbrace \pdisc: \pdisc |_{K} \in P_{0}(K), \int_{\Omega}  \pdisc =0 \ \forall K \in \mathcal{T}^{h}\right\rbrace & \; \text{if $\Gamma_{N} \cup \Gamma_{P}=\emptyset$}
  \end{array} \right. ,
\end{eqnarray*}
where $P_{0}(K)$ and $P_{1}(K)$ are respectively the spaces of constant and linear polynomials on $K$. We partition $[0,T]$ into $N$ evenly spaced non-overlapping regions $(t_{n-1}, t_n]$, $n=1,2,\dots, N$ , where $t_n-t_{n-1} = \Delta t$. For any sufficiently smooth function $v(t,x)$ we define $v^n(x) = v(t_n,x)$  and the discrete time derivative by $v_{\Delta t}^{n} := \frac{v^{n}-v^{n-1}}{\Delta t}$.

The fully discrete weak problem is: For $n = 1,2, \ldots,N$, $\dispdisc^{n} \in \dispspacedisc$, find $\fluxdisc^{n} \in \fluxspacedisc$ and $\pdisc^{n} \in \pspacedisc $ such that
\begin{subequations}
\begin{align}
&a(\dispdisc^{n},\dispdisctest)-(\pdisc^{n},\nabla \cdot \dispdisctest) =(\mathbf{f}^{n},\dispdisctest)+(\mathbf{t}_{N},\dispdisctest)_{\Gamma_{N}}
\; \forall \dispdisctest\in \dispspacedisc,
\label{eqn:weak_fulldisc_elast_mom} \\
&(\perminv \fluxdisc^{n},\fluxdisctest)-(\pdisc^{n},\nabla \cdot \fluxdisctest)
= (\mathbf{b}^{n},\fluxdisctest)
%-(p_{D},\fluxdisctest\cdot \normal)_{\Gamma_{P}}
\; \forall \fluxdisctest\in \fluxspacedisc,
\label{eqn:weak_fulldisc_flux_mom} \\
&(\nabla \cdot \dispdisctimediscn,\pdisctest)+(\nabla \cdot \fluxdisc^{n},\pdisctest) + J\left(\pdisctimediscn,\pdisctest\right)
=(g^{n},\pdisctest)  \; \forall \pdisctest\in \pspacedisc.
\label{eqn:weak_fulldisc_mass}
\end{align}
\label{eqns:weak_fulldisc_system}
\end{subequations}
The stabilization term is
\begin{equation}
J(\pcont,\pconttest)= \delta \sum_{K} \int_{\partial K \backslash \partial \Omega} h_{\partial K} [\pcont][\pconttest] \;\mbox{d}s.
\end{equation}
Here $\delta$ is a stabilization parameter that is independent of $h$ and $\Delta t $.  Here $ h_{\partial K} $ denotes the size (diameter) of an element edge in 2D or face in 3D, and $[\cdot]$ is the jump across an edge or face (taken on the interior edges only). We will see in the numerical results, section \ref{sec:results} that the convergence is not sensitive to $\delta$. The set of all elements is denoted by $K$, $ h_{\partial K} $ denotes the size of an element edge in 2D or face in 3D, and $[\cdot]$ is the jump across an edge. As an example consider $[\pdisc]$, the jump operator on the piecewise constant pressure.  The jump in pressure $[\pdisc]$ across an element or face $E$ adjoining elements $T$ and $S$ is defined such that
\begin{equation*}
(\pdisc|_{T} - \pdisc|_{S})  \normal_{E,T}=(\pdisc|_{S} - \pdisc|_{T})  \normal_{E,S}.
\end{equation*}
Here $\normal_{E,T}$ is the outward normal from element $T$, with respect to edge $E$, $\normal_{E,S}$ is the corresponding inward facing normal, and $\pdisc|_{T}$ and $\pdisc|_{S}$ denote the pressure in element $T$ and $S$, respectively.

We also assume
\begin{subequations}
\begin{align}
a(\dispdisc^{0},\dispdisctest)
&= a(\dispcont^{0},\dispdisctest)\; \forall \dispdisctest\in \dispspacedisc,
\label{eqn:a_initial} \\
J(\pdisc^{0},\pdisctest) &= J(\pcont^{0},\pdisctest)\; \forall \pdisctest\in \pspacedisctest,
\label{eqn:p_initial}
\end{align}
\label{eqns:weak_disc_initial}
\end{subequations}
where $p^{0} \in \pspace$.


 