%%%%%%%%%%%%%%%%%%%%%%%%%%%%%%%%%%%%%%%%%%%%%%%%%%%%%%%%%%%%%%%%%%%%%%%%%%%%%%%%%%%%%%%%%%%%%%%%%%%%
%                                         Introduction                                             %
%%%%%%%%%%%%%%%%%%%%%%%%%%%%%%%%%%%%%%%%%%%%%%%%%%%%%%%%%%%%%%%%%%%%%%%%%%%%%%%%%%%%%%%%%%%%%%%%%%%%

\section {Introduction}
\label{sec:intro}

In this chapter we develop a stabilized, low-order, mixed finite element method for poroelastic models of biological tissues and restrict our attention to the fully saturated, incompressible, small deformation case. Our mixed scheme uses the lowest possible approximation order: piecewise constant approximation for the pressure and piecewise linear continuous elements for the displacement and fluid flux.

To ensure stability, a mixed finite element method must satisfy the \newline Ladyzhenskaya-Babuska-Brezzi (LBB) condition. In this work we use a local pressure jump stabilization method pioneered by \cite{burman2007unified} for the study of Stokes and Darcy flows that are coupled via an interface. This approach provides the natural $H^{1}$ stability for the displacements and $H{div}$ stability for the fluid flux. The naive approach of using the stabilization of the pressure, as is done for the Darcy and Stokes equations in \cite{burman2007unified}, results in an approximation that does not converge at an optimal rate. Stabilization using the time derivative of pressure in the stabilization term is shown to be crucial for stability and optimal convergence with refinement and counterexamples are provided in Section \ref{sec:results}.

In section \ref{sec:model} we formulate the model and its continuous weak formulation and construct a fully-discrete approximation. We prove existence and uniqueness of solutions to this discrete model at each time step in section \ref{sec:stability}, provide an energy estimate over time in section \ref{sec:energy}, and derive an optimal order a-priori error estimate in section \ref{sec:apriori}. Finally in section \ref{sec:results}, we present numerical experiments to illustrate the convergence of the method and its ability to overcome pressure oscillations.







%Poroelasticity is a mixture theory in which a complex fluid-structure interaction is approximated by a superposition of the solid and fluid components.  Developments of the continuum theory can be found, for example, in \cite{coussy2004poromechanics} and \cite{boer2005trends}. Poroelastic models have been developed to study numerous geomechanical applications ranging from reservoir engineering \cite{phillips2007coupling} to earthquake fault zones \cite{white2008stabilized}. Fully saturated, incompressible poroelastic models have been proposed for a variety of biological tissues and processes, including lung parenchyma \cite{kowalczyk1993mechanical}, protein-based hydrogels embedded within cells \cite{galie2011linear}, blood flow in the beating myocardium \cite{chapelle2010poroelastic,cookson2011novel}, brain oedema and hydrocephalus \cite{li2010three,wirth2006axisymmetric}, and interstitial fluid and tissue in articular cartilage and intervertebral discs \cite{mow1980biphasic,holmes1990nonlinear,galbusera2011comparison}.

%We develop a stabilized, low-order, mixed finite element method for poroelastic models of biological tissues and restrict our attention to the fully saturated, incompressible case, and in order to simplify our presentation, we also assume small deformations. In contrast to \cite{murad1994stability,white2008stabilized,murad1994stability}, who study a reduced displacement and pressure formulation, we retain the fluid flux variable as a primary variable resulting in a three-field, displacement, fluid flux, and pressure formulation. This avoids post processing to calculate the fluid flux and material stress, and allows physically meaningful boundary conditions to be applied at the interface when modelling the interaction between a fluid and a poroelastic structure \cite{badia2009coupling}. A three-field approach can be readily extended from a Darcy to a Brinkman flow model, for which there are numerous applications in modelling biological tissues \cite{khaled2003role}. Our mixed scheme uses the lowest possible approximation order, piecewise constant approximation for the pressure and piecewise linear continuous elements for the displacement and fluid flux, since continuous pressure elements often struggle to capture the steep gradients at the interface between the high-permeable and low-permeable region. The resulting linear system is symmetric and has a block structure that is well suited for effective preconditioning.

%To ensure stability, a mixed finite element method must satisfy the Ladyzhenskaya-Babuska-Brezzi (LBB) condition. Stabilization techniques have been proposed for the Stokes equations, see e.g. \cite{elman2005finite} (chapter 5) and for Darcy flow, see e.g. \cite{bochev2006computational}. Most stabilization techniques construct a modified variational formulation in which an additional term is added to the mass balance equation. In this work we use a local pressure jump stabilization method pioneered by \cite{burman2007unified} for the study of Stokes and Darcy flows that are coupled via an interface. This approach provides the natural $H^{1}$ stability for the displacements and $H{div}$ stability for the fluid flux. % Preconditioning for the proposed method will be part of future work.

%In earlier approaches to solving the three-field problem, \cite{phillips2007coupling,phillips2007couplingtwo} developed a mixed finite element method using continuous piecewise linear approximations for displacements and mixed low-order Raviart Thomas elements for the fluid flux and pressure variables. However their method was found to be susceptible to pressure oscillations \cite{phillips2009overcoming}. To overcome these pressure oscillations, \cite{li2012discontinuous} and \cite{yi2013coupling} analysed a discontinuous and nonconforming three-field methods respectively. 

